%% LyX 2.1.2 created this file.  For more info, see http://www.lyx.org/.
%% Do not edit unless you really know what you are doing.
\documentclass[12pt,letterpaper,english]{article}
\usepackage[T1]{fontenc}
\usepackage[latin9]{inputenc}
\usepackage{color}
\usepackage{babel}
\usepackage{url}
\usepackage{amssymb}
\usepackage{cancel}
\usepackage{setspace}
\doublespacing
\usepackage[unicode=true,
 bookmarks=true,bookmarksnumbered=false,bookmarksopen=false,
 breaklinks=false,pdfborder={0 0 0},backref=false,colorlinks=true]
 {hyperref}
\hypersetup{pdftitle={Chemlogic: A Logic Programming Computer Chemistry System},
 pdfauthor={Nicholas Paun}}
\usepackage[dot]{bibtopic}

\makeatletter

%%%%%%%%%%%%%%%%%%%%%%%%%%%%%% LyX specific LaTeX commands.
\pdfpageheight\paperheight
\pdfpagewidth\paperwidth

\newcommand{\noun}[1]{\textsc{#1}}

%%%%%%%%%%%%%%%%%%%%%%%%%%%%%% User specified LaTeX commands.
\usepackage{units}

\usepackage{fancyhdr}
\pagestyle{fancy}
\usepackage{lastpage}
\lhead{}
\rhead{}
\renewcommand{\headrulewidth}{0pt}
\cfoot{\fontsize{8pt}{8pt} \selectfont 2015-04-30 --- Nicholas Paun: Chemlogic --- \thepage\ of \pageref{END}}

%\usepackage{anysize}
%\marginsize{1in}{1in}{0.5in}{0.5in}

\usepackage[top=0.625in,left=1in,bottom=0.5in,right=1in,includefoot,includehead,letterpaper]{geometry}
\topmargin 0pt
\headsep 0pt
\headheight 0pt
\usepackage{titlesec}
\titleformat{\section}
  {\normalfont\fontsize{13.5}{15}\bfseries\scshape}{\thesection}{1em}{}
\usepackage{url}

\makeatother

\begin{document}

\title{\vspace{-1in}
Chemlogic:\\
 A Logic Programming Computer Chemistry System}


\author{\emph{Second Edition}\\
Nicholas Paun}


\date{~}

\maketitle
 \thispagestyle{fancy}

\vspace{-1in}



\section*{$*$ Abstract}

\noun{Chemlogic is a logic program for computer chemistry} that performs
stoichiometric calculations, balances and completes equations, and
converts between formulas and names. The program has applications
in education, particularly as a study tool. Features are implemented
using a chemical information database, linear equation solver, and
grammatical rules. Guidance is provided for resolving errors in user
input. Chemlogic is available on Android and the Web.


\section*{Background}

\noun{In high school Chemistry}, students learn to work with chemical
compounds, equations and stoichiometric calculations. Algorithms were
researched, adapted to chemistry problems and implemented in Prolog,
to develop a program that could be used by both students and teachers.


\section*{Design and Implementation}

\noun{User input is parsed} into a form that can be easily manipulated
and transformed. A parser recognizes a formal grammar describing valid
user input. In Prolog, parsers are implemented using DCGs (Definite
Clause Grammars), which provide a simplified syntax for creating logical
clauses that process a grammar using difference lists, an efficient
representation. \cite{triska:dcg}

Internal representations must be created for the parsed input. Chemlogic
uses a pseudo-Abstract Syntax Tree to record the structure of an equation,
as well as lists containing useful information (e.g. the elements
contained in an equation).\\
\\
\noun{The reaction type analysis module} identifies common types of
chemical reactions, predicts whether a reaction will take place, and
completes chemical equations for common reactions, given the reactants.
These features are implemented by matching the Abstract Syntax Tree
generated by parsing an equation, against a pattern representing the
structure of a certain chemical reaction type. In Prolog, this matching
is performed by syntactic unification, which is the process of combining
two structures by replacing variables with constant terms, so as to
make the two structures equal. 

Once the reaction type is determined, a prediction can be made as
to whether the reaction will occur by comparing the relative positions
in the reactivity series of the elements involved. The reaction type
and the reaction prediction can then be displayed by the user interfaces,
as additional information for the entered chemical equation.

To complete a chemical equation, the complete formulas of the product
compounds must be calculated, given the arrangements of elements and
ions determined by reaction pattern. This process is performed using
rules specific to the type of compound being formed by the reaction.
The complete formulas are then substituted into the equation structure,
to produce a complete chemical equation. This structure can then be
used as input by other modules, such as the chemical equation balancer.\\
\noun{}\\
\noun{Balancing of chemical equations} is usually performed by inspection.
\cite{sandner2008bc} This process cannot easily be used in a program
because it is unsystematic. Instead, Chemlogic uses an efficient algorithm,
representing a chemical equation as a system of linear equations.
One linear equation is created for every element in a chemical equation,
with the number of occurrences of the element in each formula representing
a coefficient, multiplied by an unknown (the chemical equation coefficient).
\cite{tuckerman:nyu}

These systems are commonly solved by converting them to a matrix and
applying Gaussian elimination. \cite{eigenstate:chembalancerjs} In
Chemlogic, a matrix is produced from structures created by the parser
and lookup tables. The matrix is converted into a system of linear
equations, which is then provided to the built-in CLP(q) facility,
which can solve constraints over rational numbers. \cite{holzbaur:clpq}\\
\noun{}\\
\noun{Stoichiometry is an important new feature} of Chemlogic. The
process of performing stoichiometric calculations begins by parsing
a chemical equation that includes quantities for some compounds. Whenever
a quantity is parsed, the number of significant figures in the value
is determined and recorded in the quantity structure. A list of queries
for quantities to be determined is also parsed. 

The Abstract Syntax Tree for the given chemical equation is then traversed
to determine the number of known quantities. If more than one quantity
is known, the limiting reactant must be determined and used to calculate
the results. This is performed by converting each known reactant quantity
to moles (the standard unit for the amount of a chemical substance),
then dividing by the corresponding coefficient of the reactant in
the equation. The limiting reactant always has the lowest value for
this ratio. If only one quantity is known, it must, logically, be
used to perform all the calculations.\noun{ }

Once the correct quantity and compound to use\noun{ }as input for
the calculation is determined, the list of queries is traversed, and
the input quantity is converted to each requested output quantity.
This calculation is performed according to the same method that is
typically used when performing stoichiometric calculations manually.
Unit conversions are performed by simple calculation rules, which
take into account the maximum number of significant figures that can
be yielded by the calculation.\noun{}\\
\noun{}\\
\noun{Syntax errors} cannot simply cause a program to fail --- clear
identification and explanation of an error is necessary. When a predicate
that must succeed for a given input to be valid fails, a syntax error
exception is thrown, containing a code name for the error and the
remaining unparsed input (within the tail). The exception handler
attempts to localize the error by highlighting only the erroneous
part of the tail, using rules specific to the type of the first character.\\
\noun{}\\
\noun{The Android App was developed} to make Chemlogic more readily
available as a study tool. The App consists of a user-interface, implemented
in Java, using the Android APIs, and a package consisting of the cross-compiled
code of Chemlogic and its dependencies. The user interface communicates
with Chemlogic through a UNIX pipe. Upon receiving the solution from
Chemlogic, the interface renders the formatting of the result and
displays it. 

The Chemlogic package contains a copy of the Chemlogic command-line
interface, compiled as a stand-alone application with an embedded
Prolog interpreter in its binary, some necessary Linux libraries and
an initialization script. When loaded, the App executes the initialization
script, which starts Chemlogic by running the dynamic linker to locate
and link Chemlogic with the provided libraries.


\section*{Discussion}

\noun{Prolog was chosen} as the language for Chemlogic because its
features make it well suited to writing programs of this type in a
simple and efficient way. The built-in Definite Clause Grammars syntax
allows a programmer to implement advanced parsers using a very simple
syntax, and strong support for metaprogramming allows syntax and code
to be simplified in powerful ways. Using a logic programming language,
such as Prolog, enables the programmer to describe the results, instead
of the process. \cite{wielemaker:2011:tplp}\\
\\
\noun{Performance was analyzed} in Chemlogic by counting inferences
(provided by \texttt{time/1}) used by different algorithms for various
problem sizes. Algorithms were compared on their fixed inferences
(intercept), inferences per item (slope) and to ensure that their
complexities were not exponential.\noun{}\\
\noun{}\\
\noun{Further research and development} --- The parsers currently
implemented in Chemlogic process user input directly, as character
lists, without a tokenization process. Using new features available
in version 7 of SWI-Prolog, a simple and fast procedure could be implemented
to separate user input into space-delimited parts, \cite{swi:doublequote}
significantly reducing the number of tests and operations required
to perform parsing. 

Developing a feature to automatically generate and mark random chemistry
problems for student review and test creation would be an important
extension to Chemlogic. With the addition of reaction type analysis,
the preliminary development for this feature has been completed. The
reaction type, equation, formula and name grammars in Chemlogic could
be extended to produce random valid structures, in addition to simply
recognizing and converting them.


\section*{Conclusions\enlargethispage{\baselineskip}}

\noun{Each module of Chemlogic} is designed to transform the standardized
Abstract Syntax Trees generated by the parsers, allowing for modules
to be composed together, each adding a piece of functionality as structures
are passed from predicate to predicate. In this way, existing modules
can make use of new functionality, new features can be implemented
based on existing calculations and structures, and modules can be
connected to as part of integrated user interfaces.\\
\\
\noun{Chemlogic was successfully extended} to implement new features,
based upon the existing modular structure, and can now perform stoichiometric
calculations, convert chemical quantities between units, and complete
chemical reactions. The balancing and formula to name conversion features
were further developed and made available in a mobile application
for Android.\label{END}

\appendix
\pagebreak{}

\pagenumbering{roman}
\cfoot{\fontsize{8pt}{8pt} \selectfont 2015-04-30 --- Nicholas Paun: Chemlogic --- \thepage\ of \pageref{APPEND}} 


\section*{Earlier Work}

The original version of Chemlogic was an experimental program I developed
in 2012, to simplify my Science homework. 

Based on these experiments, the project was extended in 2013 and 2014.
The program was rewritten in Prolog, new algorithms were studied and
adapted and many new features were implemented. The current modular
design and structure of Chemlogic was implemented in Version 1, which
supported balancing chemical equations, converting chemical names
to formulas, and vice-versa. This version was presented at the 2014
CWSF in Windsor, where it received a Bronze Excellence Award.

In 2014 and 2015, Chemlogic was further developed to significantly
expand the project, by continuing to improve existing features, adding
a new user interface and adding support for higher levels of high
school Chemistry. Version 2 of Chemlogic introduced the Android user
interface and support for stoichiometric calculations, chemical unit
conversions and reaction type analysis.


\section*{Acknowledgments}

I would like to thank the many people who gave advice and helped with
the project.

I am particularly grateful for the invaluable assistance provided
by Dr. Peter Tchir, now retired, my Physics, Chemistry and Computer
Science teacher. His help and advice, especially with algorithms,
and his support for my Computer Science projects helped make this
program possible.

I would also like to thank Mr. Jason Peil for his assistance in designing
the display, and Mr. Greg Osadchuk for his input and assistance regarding
the visual presentation.


\section*{Obtaining Chemlogic / Contact}

Nicholas Paun <\url{np@icebergsystems.ca}>

Chemlogic is open-source software. A copy of the program and additional
information is available at \url{http://icebergsys.ca/chemlogic}


\section*{References}

\begin{btSect}[plainurl]{chemlogic}
\btPrintAll
\end{btSect}
\label{APPEND}
\end{document}
